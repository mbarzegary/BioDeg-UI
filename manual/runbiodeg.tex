%After compiling \biodeg{} as described in Section~\ref{sec:installation}, we have now two executables --- \verb|/build/release/real/dftfe| and \verb|/build/release/complex/dftfe|. The \verb|/build/release/real/dftfe| executable, which uses real data-structures is sufficient for fully non-periodic problems. The executable can also be used for periodic and semi-periodic problems involving a Gamma point calculation. On the other hand the \verb|/build/release/complex/dftfe| executable, which uses complex data-structures is required for periodic and semi-periodic problems with multiple k point sampling for Brillouin zone integration. These executables are to be used as follows-- for a serial run use
%\begin{verbatim}
%  ./dftfe parameterFile.prm
%\end{verbatim}
%or, for a parallel run use
%\begin{verbatim}
%  mpirun -n N ./dftfe parameterFile.prm
%\end{verbatim}
%to run with N processors. 

\subsection{Configuring the simulation}


\subsubsection{Example 1}\label{sec:example1}

%Let us consider the first example given in the folder
%\verb|/dftfe/demo/ex1|, where we compute the 

\subsubsection{Example 2}\label{sec:example2}
