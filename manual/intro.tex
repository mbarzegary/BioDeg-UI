\biodeg{} is an open source software written in FreeFEM (a domain-specific language for finite element programming), C++, and Python for modeling the degradation of metallic biomaterials and simulating the biodegradation behavior of medical devices, implants and scaffolds in corrosion experiments. It can handle any geometry of desire and supports parallel computing to simulate large scale models.

\subsection{Authors}
\label{sec:authors}
\biodeg{} is developed by the \href{http://www.biomech.ulg.ac.be/}{Biomechanics Research group at KU Leuven and University of Liege}. The code is currently maintained by its principal developer, 
who manage the development of the mathematical models and the core functionalities. 

\paragraph*{Principal developer}
\begin{itemize}
	\item Mojtaba Barzegari (University of Leuven, Belgium)
\end{itemize}

\paragraph*{Previous Contributors}
\begin{itemize}
	\item Yann Guyot (University of Liege, Belgium)
	\item Piotr Bajger (University of Oxford, UK)
\end{itemize}

\paragraph*{Mentor}
\begin{itemize}
	\item Liesbet Geris (University of Leuven, Belgium)
\end{itemize}

\paragraph*{Chemist contributors}
(who has helped to validate the models)
\begin{itemize}
	\item Sviatlana V. Lamaka (Helmholtz-Zentrum Hereon, Gremany)
	\item Di Mei (Zhengzhou University, China)
	\item Cheng Wang (Helmholtz-Zentrum Hereon, Gremany)
\end{itemize}

\subsection{Acknowledgments}
The development of \biodeg{} open source code is financially supported by the Prosperos project, funded by the Interreg VA Flanders – The Netherlands program, CCI grant no. 2014TC16RFCB046 and by the Fund for Scientific Research Flanders (FWO), grant G085018N. The developers also acknowledge support from the European Research Council under the European Union's Horizon 2020 research and innovation programmen, ERC CoG 772418.

\subsection{Referencing \biodeg{}}

Please refer to \href{https://github.com/mbarzegary/BioDeg}{\biodeg{} repository}, section "Publications and referencing" to properly cite the use of 
\biodeg{} in your scientific work. 
