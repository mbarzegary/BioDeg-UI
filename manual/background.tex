You may refer to the following articles for a background of the methods and algorithms implemented in \biodeg:

\begin{enumerate}
\item
M. Barzegari, D. Mei, S.V. Lamaka, L. Geris, \href{https://doi.org/10.1016/j.corsci.2021.109674}{Computational modeling of degradation process of biodegradable magnesium biomaterials}, \emph{Corrosion Science}, 190, 109674, 2021.
\item
M. Barzegari, L. Geris, \href{https://doi.org/10.1177/10943420211045939}{Highly scalable numerical simulation of coupled reaction–diffusion systems with moving interfaces}, \emph{The International Journal of High Performance Computing Applications}, 2021.
\end{enumerate}

\noindent In addition, below are some useful references on finite element method and some online resources that provide a background of finite elements and their application to the solution of partial differential equations.

\begin{enumerate}
\item
H.P. Langtangen, K.A. Mardal, \href{https://link.springer.com/book/10.1007/978-3-030-23788-2}{Introduction to Numerical Methods for Variational Problems}, Springer, 2019: a nice book to understand variational formulation required for finite element computations.
\item
\href{https://github.com/TuxRiders/numerical-computing-intro}{An introduction to applied numerical computing}: A set of Jupyter notebooks on various aspects of numerical methods, including the notebooks covering the topics of the book mentioned above.
\item
\href{https://www.youtube.com/playlist?list=PL6fjYEpJFi7UMDXtNiaF3eLlOKAM8Lrkf}{YouTube video series} discussing different numerical methods concepts including the finite difference and finite element methods.
\end{enumerate}



