\subsection{Geometry and mesh parameters}
\label{parameters:mesh}

\begin{itemize}

\item {\it Parameter name:} {\tt import\_mesh}
\phantomsection
\label{parameters:import_mesh}

\index[prmindex]{import\_mesh}
\index[prmindexfull]{Geometry and mesh!import mesh}

{\it Default:} true (1)

{\it Description:} [Standard] Boolean parameter specifying whether an external mesh file is imported or a container box, as well as a cubic scaffold, would be created on the fly for simulation.

{\it Possible values:} A boolean value (1 or 0)

\item {\it Parameter name:} {\tt mesh\_file}
\phantomsection
\label{parameters:mesh_file}

\index[prmindex]{mesh\_file}
\index[prmindexfull]{Geometry and mesh!mesh file}

{\it Default:} Should be provided

{\it Description:} [Standard] Path to the input mesh file (in MEDIT {\tt .mesh} format), which can be either absolute or relative. Is relevant only if parameter {\tt import\_mesh} is set to TRUE.

{\it Possible values:} Any string value 


\item {\it Parameter name:} {\tt label\_scaffold}
\phantomsection
\label{parameters:label_scaffold}

\index[prmindex]{label\_scaffold}
\index[prmindexfull]{Geometry and mesh!label scaffold}

{\it Default:} 1

{\it Description:} [Standard] The label of the (volume) region which is supposed to be scaffold in the input mesh (can be viewed in programs like GMSH before importing into \biodeg{}). 

{\it Possible values:} Any positive integer value 


\item {\it Parameter name:} {\tt label\_medium}
\phantomsection
\label{parameters:label_medium}

\index[prmindex]{label\_medium}
\index[prmindexfull]{Geometry and mesh!label medium}

{\it Default:} 2 

{\it Description:} [Standard] The label of the (volume) region which is supposed to be the medium (electrolyte) in the input mesh.

{\it Possible values:} Any positive integer value 


\item {\it Parameter name:} {\tt label\_wall}
\phantomsection
\label{parameters:label_wall}

\index[prmindex]{label\_wall}
\index[prmindexfull]{Geometry and mesh!label wall}

{\it Default:} 3

{\it Description:} [Advanced] The label of the surface in the input mesh to be assigned as wall (no-slip boundary condition) in the fluid flow simulations.

{\it Possible values:} Any positive integer value 


\item {\it Parameter name:} {\tt label\_inlet}
\phantomsection
\label{parameters:label_inlet}

\index[prmindex]{label\_inlet}
\index[prmindexfull]{Geometry and mesh!label inlet}

{\it Default:} 4

{\it Description:} [Advanced] The label of the surface in the input mesh to be assigned as flow inlet (constant velocity boundary condition) in the fluid flow simulations.

{\it Possible values:} Any positive integer value 


\item {\it Parameter name:} {\tt label\_outlet}
\phantomsection
\label{parameters:label_outlet}

\index[prmindex]{label\_outlet}
\index[prmindexfull]{Geometry and mesh!label outlet}

{\it Default:} 5

{\it Description:} [Advanced] The label of the surface in the input mesh to be assigned as flow outlet (zero pressure boundary condition) in the fluid flow simulations.

{\it Possible values:} Any positive integer value 


\item {\it Parameter name:} {\tt box\_length}
\phantomsection
\label{parameters:box_length}

\index[prmindex]{box\_length}
\index[prmindexfull]{Geometry and mesh!box length}

{\it Default:} 20.0

{\it Description:} [Standard] In case of {\tt import\_mesh} being FALSE, specifies the length of the container box (for the electrolyte) in mm.

{\it Possible values:} Any positive floating-point number 


\item {\it Parameter name:} {\tt cube\_size\_x}
\phantomsection
\label{parameters:cube_size_x}

\index[prmindex]{cube\_size\_x}
\index[prmindexfull]{Geometry and mesh!cube size x}

{\it Default:} 13.0

{\it Description:} [Standard] In case of {\tt import\_mesh} being FALSE, specifies the length of the scaffold cuboid along the x axis in mm.

{\it Possible values:} Any positive floating-point number 


\item {\it Parameter name:} {\tt cube\_size\_y}
\phantomsection
\label{parameters:cube_size_y}

\index[prmindex]{cube\_size\_y}
\index[prmindexfull]{Geometry and mesh!cube size y}

{\it Default:} 13.0

{\it Description:} [Standard] In case of {\tt import\_mesh} being FALSE, specifies the length of the scaffold cuboid along the y axis in mm.

{\it Possible values:} Any positive floating-point number 


\item {\it Parameter name:} {\tt cube\_size\_z}
\phantomsection
\label{parameters:cube_size_z}

\index[prmindex]{cube\_size\_z}
\index[prmindexfull]{Geometry and mesh!cube size z}

{\it Default:} 4.0

{\it Description:} [Standard] In case of {\tt import\_mesh} being FALSE, specifies the length of the scaffold cuboid along the z axis in mm.


{\it Possible values:} Any positive floating-point number 


\item {\it Parameter name:} {\tt mesh\_size}
\phantomsection
\label{parameters:mesh_size}

\index[prmindex]{mesh\_size}
\index[prmindexfull]{Geometry and mesh!mesh size}

{\it Default:} 32

{\it Description:} [Standard] Number of elements on each edge of the container box, so a higher number means a finer mesh. The mesh size of the cuboid will be adjusted accordingly or can be adaptively refined by setting parameter {\tt refine\_initial\_mesh} to TRUE.

{\it Possible values:} Any positive integer number 


\item {\it Parameter name:} {\tt refine\_initial\_mesh}
\phantomsection
\label{parameters:refine_initial_mesh}

\index[prmindex]{refine\_initial\_mesh}
\index[prmindexfull]{Geometry and mesh!refine initial mesh}

{\it Default:} false (0)

{\it Description:} [Advanced] A boolean parameter specifying if the mesh (no matter if imported or generated) should be adaptively refined on the metal-medium interface (corrosion surface). This affects the beginning of the simulation only (on the initial mesh).

{\it Possible values:} A boolean value (1 or 0)


\item {\it Parameter name:} {\tt mshmet\_error}
\phantomsection
\label{parameters:mshmet_error}

\index[prmindex]{mshmet\_error}
\index[prmindexfull]{Geometry and mesh!mshmet error}

{\it Default:} 0.01

{\it Description:} [Advanced] Since the open-source tool {\tt mshmet} is used for creating a metric for refining the mesh on the level set signed distance function, a tolerance should be specified for it. A lower value results to a finer mesh.

{\it Possible values:} Any floating-point number


\item {\it Parameter name:} {\tt mesh\_size\_min}
\phantomsection
\label{parameters:mesh_size_min}

\index[prmindex]{mesh\_size\_min}
\index[prmindexfull]{Geometry and mesh!mesh size min}

{\it Default:} 0.04

{\it Description:} [Advanced] Specifies the smallest element size to be passed to the {\tt tetgen} mesh generator for refining the initial mesh.

{\it Possible values:} Any floating-point number


\item {\it Parameter name:} {\tt mesh\_size\_max}
\phantomsection
\label{parameters:mesh_size_max}

\index[prmindex]{mesh\_size\_max}
\index[prmindexfull]{Geometry and mesh!mesh size max}

{\it Default:} 0.8

{\it Description:} [Advanced] Specifies the largest element size to be passed to the {\tt tetgen} mesh generator for refining the initial mesh.

{\it Possible values:} Any floating-point number

\end{itemize}



\subsection{Materials and boundary conditions parameters}
\label{parameters:material}

\begin{itemize}
\item {\it Parameter name:} {\tt material\_density}
\phantomsection
\label{parameters:material_density}

\index[prmindex]{material\_density}
\index[prmindexfull]{Materials and boundary conditions!material density}

{\it Default:} 1.735e-3

{\it Description:} [Standard] The density of the metallic material. The default value is the density of magnesium.

{\it Possible values:} Any floating-point number


\item {\it Parameter name:} {\tt film\_density}
\phantomsection
\label{parameters:film_density}

\index[prmindex]{film\_density}
\index[prmindexfull]{Materials and boundary conditions!film density}

{\it Default:} 2.3446e-3

{\it Description:} [Standard] The density of the protective film that forms on the corrosion surface. The default value is the density of magnesium hydroxide.

{\it Possible values:} Any floating-point number


\item {\it Parameter name:} {\tt material\_satur}
\phantomsection
\label{parameters:material_satur}

\index[prmindex]{material\_satur}
\index[prmindexfull]{Materials and boundary conditions!material satur}

{\it Default:} 0.134e-3

{\it Description:} [Advanced] The saturation concentration at which the metallic material (here, the ions) saturates through the medium. The default value is defined for magnesium ions.

{\it Possible values:} Any floating-point number


\item {\it Parameter name:} {\tt material\_eps}
\phantomsection
\label{parameters:material_eps}

\index[prmindex]{material\_eps}
\index[prmindexfull]{Materials and boundary conditions!material eps}

{\it Default:} 0.55

{\it Description:} [Advanced] The porosity of the formed protective layer in the range [0, 1].

{\it Possible values:} Any floating-point number between 0 and 1


\item {\it Parameter name:} {\tt material\_tau}
\phantomsection
\label{parameters:material_tau}

\index[prmindex]{material\_tau}
\index[prmindexfull]{Materials and boundary conditions!material tau}

{\it Default:} 1.0

{\it Description:} [Advanced] The tortuosity of the formed protective layer.

{\it Possible values:} Any floating-point number


\item {\it Parameter name:} {\tt d\_mg}
\phantomsection
\label{parameters:d_mg}

\index[prmindex]{d\_mg}
\index[prmindexfull]{Materials and boundary conditions!d mg}

{\it Default:} 0.05

{\it Description:} [Standard] 

{\it Possible values:} The diffusion coefficient of the metallic ions transport. This parameter is one of the most effective ones on the rate of degradation.


\item {\it Parameter name:} {\tt d\_cl}
\phantomsection
\label{parameters:d_cl}

\index[prmindex]{d\_cl}
\index[prmindexfull]{Materials and boundary conditions!d cl}

{\it Default:} 0.05

{\it Description:} [Standard] The diffusion coefficient of the chloride ions transport

{\it Possible values:} Any floating-point number


\item {\it Parameter name:} {\tt d\_oh}
\phantomsection
\label{parameters:d_oh}

\index[prmindex]{d\_oh}
\index[prmindexfull]{Materials and boundary conditions!d oh}

{\it Default:} 25.2

{\it Description:} [Standard] The diffusion coefficient of the hydroxide ions transport

{\it Possible values:} Any floating-point number


\item {\it Parameter name:} {\tt k1}
\phantomsection
\label{parameters:k1}

\index[prmindex]{k1}
\index[prmindexfull]{Materials and boundary conditions!k1}

{\it Default:} 7.0

{\it Description:} [Standard] The reaction rate at which the chemical reaction of the protective film formation occurs.

{\it Possible values:} Any floating-point number


\item {\it Parameter name:} {\tt k2}
\phantomsection
\label{parameters:k2}

\index[prmindex]{k2}
\index[prmindexfull]{Materials and boundary conditions!k2}

{\it Default:} 1e15

{\it Description:} [Standard] The reaction rate at which the chemical reaction of the protective film dissolution occurs.

{\it Possible values:} Any floating-point number


\item {\it Parameter name:} {\tt fluid\_nu}
\phantomsection
\label{parameters:fluid_nu}

\index[prmindex]{fluid\_nu}
\index[prmindexfull]{Materials and boundary conditions!fluid nu}

{\it Default:} 0.85

{\it Description:} [Advanced] The effective viscosity of the fluid used to simulate hydrodynamics conditions.

{\it Possible values:} Any floating-point number


\item {\it Parameter name:} {\tt fluid\_in\_x}
\phantomsection
\label{parameters:fluid_in_x}

\index[prmindex]{fluid\_in\_x}
\index[prmindexfull]{Materials and boundary conditions!fluid in x}

{\it Default:} 0.1

{\it Description:} [Advanced] The X component of the fluid velocity defined on the inlet (see {\tt label\_inlet}) as the boundary condition for the fluid flow.

{\it Possible values:} Any floating-point number


\item {\it Parameter name:} {\tt fluid\_in\_y}
\phantomsection
\label{parameters:fluid_in_y}

\index[prmindex]{fluid\_in\_y}
\index[prmindexfull]{Materials and boundary conditions!fluid in y}

{\it Default:} 0

{\it Description:} [Advanced] The Y component of the fluid velocity defined on the inlet (see {\tt label\_inlet}) as the boundary condition for the fluid flow.

{\it Possible values:} Any floating-point number


\item {\it Parameter name:} {\tt fluid\_in\_z}
\phantomsection
\label{parameters:fluid_in_z}

\index[prmindex]{fluid\_in\_z}
\index[prmindexfull]{Materials and boundary conditions!fluid in z}

{\it Default:} 0

{\it Description:} [Advanced] The Z component of the fluid velocity defined on the inlet (see {\tt label\_inlet}) as the boundary condition for the fluid flow.

{\it Possible values:} Any floating-point number


\item {\it Parameter name:} {\tt initial\_cl}
\phantomsection
\label{parameters:initial_cl}

\index[prmindex]{initial\_cl}
\index[prmindexfull]{Materials and boundary conditions!initial cl}

{\it Default:} 5.175e-6

{\it Description:} [Standard] Initial concentration of chloride ions in the medium.

{\it Possible values:} Any floating-point number


\item {\it Parameter name:} {\tt initial\_oh}
\phantomsection
\label{parameters:initial_oh}

\index[prmindex]{initial\_oh}
\index[prmindexfull]{Materials and boundary conditions!initial oh}

{\it Default:} 1e-7

{\it Description:} [Standard] Initial concentration of hydroxide ions in the medium, used for computing pH. The default value indicates a pH of 7.

{\it Possible values:} Any floating-point number


\end{itemize}



\subsection{Solver parameters}
\label{parameters:sovler}

\begin{itemize}
\item {\it Parameter name:} {\tt solve\_mg}
\phantomsection
\label{parameters:solve_mg}

\index[prmindex]{solve\_mg}
\index[prmindexfull]{Solver!solve mg}

{\it Default:} true (1)

{\it Description:} [Standard] Boolean parameter indicating whether the equation for material dissolution and ions release should be solved or not. This equation is the most essential equation and in most use-cases should be solved.

{\it Possible values:} Any boolean value (1 or 0)


\item {\it Parameter name:} {\tt solve\_film}
\phantomsection
\label{parameters:solve_film}

\index[prmindex]{solve\_film}
\index[prmindexfull]{Solver!solve film}

{\it Default:} true (1)

{\it Description:} [Standard] Boolean parameter indicating whether the equation for the protective film formation should be solved or not.

{\it Possible values:} Any boolean value (1 or 0)


\item {\it Parameter name:} {\tt solve\_cl}
\phantomsection
\label{parameters:solve_cl}

\index[prmindex]{solve\_cl}
\index[prmindexfull]{Solver!solve cl}

{\it Default:} true (1)

{\it Description:} [Standard] Boolean parameter indicating whether the equation for the transport of chloride ions should be solved or not.

{\it Possible values:} Any boolean value (1 or 0)


\item {\it Parameter name:} {\tt solve\_oh}
\phantomsection
\label{parameters:solve_oh}

\index[prmindex]{solve\_oh}
\index[prmindexfull]{Solver!solve oh}

{\it Default:} true (1)

{\it Description:} [Standard] Boolean parameter indicating whether the equation for the transport of hydroxide ions should be solved or not. This equation is essential for calculating the pH changes, if desired.

{\it Possible values:} Any boolean value (1 or 0)


\item {\it Parameter name:} {\tt solve\_ls}
\phantomsection
\label{parameters:solve_ls}

\index[prmindex]{solve\_ls}
\index[prmindexfull]{Solver!solve ls}

{\it Default:} true (1)

{\it Description:} [Standard] Boolean parameter indicating whether the level set surface tracking equation should be solved or not. Surface tracking is essential for computing mass loss.

{\it Possible values:} Any boolean value (1 or 0)


\item {\it Parameter name:} {\tt solve\_fluid}
\phantomsection
\label{parameters:solve_fluid}

\index[prmindex]{solve\_fluid}
\index[prmindexfull]{Solver!solve fluid}

{\it Default:} false (0)

{\it Description:} [Standard] Boolean parameter indicating whether the fluid flow equation should be solved or not.

{\it Possible values:} Any boolean value (1 or 0)


\item {\it Parameter name:} {\tt solve\_full\_ns}
\phantomsection
\label{parameters:solve_full_ns}

\index[prmindex]{solve\_full\_ns}
\index[prmindexfull]{Solver!solve full ns}

{\it Default:} true (1)

{\it Description:} [Advanced] Boolean parameter specifying which fluid flow equation to solve: the full transient Navier-Stokes equations or a steady-state Stokes equation. A true value (1) results in \biodeg{} solving the former equation.

{\it Possible values:} Any boolean value (1 or 0)


\item {\it Parameter name:} {\tt write\_fluid\_output}
\phantomsection
\label{parameters:write_fluid_output}

\index[prmindex]{write\_fluid\_output}
\index[prmindexfull]{Solver!write fluid output}

{\it Default:} true (1)

{\it Description:} [Advanced] Boolean parameter specifying whether the fluid flow quantities (velocity field components and pressure) should be saved in the output VTK file or not. Requires {\tt write\_vtk} to be TRUE.

{\it Possible values:} Any boolean value (1 or 0)


\item {\it Parameter name:} {\tt solve\_fluid\_each}
\phantomsection
\label{parameters:solve_fluid_each}

\index[prmindex]{solve\_fluid\_each}
\index[prmindexfull]{Solver!solve fluid each}

{\it Default:} 10

{\it Description:} [Advanced] Determines the number of time steps to skip before solving the specified fluid flow equation. For example, if the value is set to 10 (default value), the fluid equation gets solved in time steps 1, 11, 21, \ldots.

{\it Possible values:} Any positive integer number


\item {\it Parameter name:} {\tt time\_step}
\phantomsection
\label{parameters:time_step}

\index[prmindex]{time\_step}
\index[prmindexfull]{Solver!time step}

{\it Default:} 0.025

{\it Description:} [Advanced] The time step value of the simulations.

{\it Possible values:} Any floating-point number


\item {\it Parameter name:} {\tt final\_time}
\phantomsection
\label{parameters:final_time}

\index[prmindex]{final\_time}
\index[prmindexfull]{Solver!final time}

{\it Default:} 21.0

{\it Description:} [Standard] The final simulation time, meaning the duration of interest for the biodegradation model.

{\it Possible values:} Any floating-point number


\item {\it Parameter name:} {\tt do\_redistance}
\phantomsection
\label{parameters:do_redistance}

\index[prmindex]{do\_redistance}
\index[prmindexfull]{Solver!do redistance}

{\it Default:} true (1)

{\it Description:} [Advanced] Indicates whether the redistancing of the level-est distance function should be done or not. Please refer to the theory guides to see how this affects the simulation.

{\it Possible values:} Any boolean value (1 or 0)


\item {\it Parameter name:} {\tt redistance\_time}
\phantomsection
\label{parameters:redistance_time}

\index[prmindex]{redistance\_time}
\index[prmindexfull]{Solver!redistance time}

{\it Default:} 1.0

{\it Description:} [Advanced] In case {\tt do\_redistance} is true, this parameter indicates the interval between each level set function re-initialization.

{\it Possible values:} Any floating-point number

\end{itemize}


\subsection{Output parameters}
\label{parameters:output}

\begin{itemize}
\item {\it Parameter name:} {\tt text\_output\_file}
\phantomsection
\label{parameters:text_output_file}

\index[prmindex]{text\_output\_file}
\index[prmindexfull]{Output!text output file}

{\it Default:} "output/result.txt"

{\it Description:} [Standard] Path to the text file in which text output, like the mass loss and evolved hydrogen production, is written over time. The path can be relative or absolute.

{\it Possible values:} Any string value referring to a valid path


\item {\it Parameter name:} {\tt write\_vtk}
\phantomsection
\label{parameters:write_vtk}

\index[prmindex]{write\_vtk}
\index[prmindexfull]{Output!write vtk}

{\it Default:} true (1)

{\it Description:} [Standard] Indicates whether VTK output (in the VTU format) should be written or not. This is required for further post-processing of the results using ParaView.

{\it Possible values:} Any boolean value (1 or 0)


\item {\it Parameter name:} {\tt vtk\_output\_name}
\phantomsection
\label{parameters:vtk_output_name}

\index[prmindex]{vtk\_output\_name}
\index[prmindexfull]{Output!vtk output name}

{\it Default:} "output/output"

{\it Description:} [Standard] The naming scheme for the VTK output. This is mainly for the PVD file, and the final name of the rest of the files will be determined by the number of employed MPI computing nodes and the time step. The number of VTU files saved per time step equals the number of employed MPI nodes.

{\it Possible values:} Any string value referring to a valid path


\item {\it Parameter name:} {\tt save\_each}
\phantomsection
\label{parameters:save_each}

\index[prmindex]{save\_each}
\index[prmindexfull]{Output!save each}

{\it Default:} 0.25

{\it Description:} [Standard] The interval of saving results, text and VTK (if selected), to disk. 

{\it Possible values:} Any floating-point number


\item {\it Parameter name:} {\tt save\_last\_state}
\phantomsection
\label{parameters:save_last_state}

\index[prmindex]{save\_last\_state}
\index[prmindexfull]{Output!save last state}

{\it Default:} true (1)

{\it Description:} [Standard] Indicates whether the last state of the system should be saved or not. The last state will be always saved on a global (non-partitioned) mesh, meaning that it will be a single VTU file, in contrast to a normal save in which the number of written files equals to the number of computing nodes.

{\it Possible values:} Any boolean value (1 or 0)


\item {\it Parameter name:} {\tt output\_per\_area}
\phantomsection
\label{parameters:output_per_area}

\index[prmindex]{output\_per\_area}
\index[prmindexfull]{Output!output per area}

{\it Default:} false (0)

{\it Description:} [Advanced] Indicated whether the side hydrogen evolution results should be computed per unit area of the exposed surface.

{\it Possible values:} Any boolean value (1 or 0)


\item {\it Parameter name:} {\tt save\_multiplier}
\phantomsection
\label{parameters:save_multiplier}

\index[prmindex]{save\_multiplier}
\index[prmindexfull]{Output!save multiplier}

{\it Default:} 1.0

{\it Description:} [Advanced] In the case of symmetrical conditions, this parameter can be used to multiply the obtained quantitative results to have an easier comparison with experimental results. A value of 1.0 indicates no multiplication.

{\it Possible values:} Any floating-point number


\item {\it Parameter name:} {\tt export\_scaffold}
\phantomsection
\label{parameters:export_scaffold}

\index[prmindex]{export\_scaffold}
\index[prmindexfull]{Output!export scaffold}

{\it Default:} false (0)

{\it Description:} [Advanced] Indicates if the degrading material should be exported as a single entity for further analysis like in a structural mechanics simulation. It works based on Mmg level set meshing.

{\it Possible values:} Any boolean value (1 or 0)


\item {\it Parameter name:} {\tt export\_scaffold\_each}
\phantomsection
\label{parameters:export_scaffold_each}

\index[prmindex]{export\_scaffold\_each}
\index[prmindexfull]{Output!export scaffold each}

{\it Default:} 1.0

{\it Description:} [Advanced] In case {\tt export\_scaffold} is true, this parameter determines the interval of saving the material mesh to the disk.

{\it Possible values:} Any floating-point number


\item {\it Parameter name:} {\tt export\_scaffold\_volume}
\phantomsection
\label{parameters:export_scaffold_volume}

\index[prmindex]{export\_scaffold\_volume}
\index[prmindexfull]{Output!export scaffold volume}

{\it Default:} true (1)

{\it Description:} [Advanced] In case {\tt export\_scaffold} is true, this parameter indicates whether a volume mesh should be saved as output or not.

{\it Possible values:} Any boolean value (1 or 0)


\item {\it Parameter name:} {\tt export\_scaffold\_surface}
\phantomsection
\label{parameters:export_scaffold_surface}

\index[prmindex]{export\_scaffold\_surface}
\index[prmindexfull]{Output!export scaffold surface}

{\it Default:} true (1)

{\it Description:} [Advanced] In case {\tt export\_scaffold} is true, this parameter indicates whether a surface mesh should be saved as output or not.

{\it Possible values:} Any boolean value (1 or 0)


\item {\it Parameter name:} {\tt save\_initial\_mesh}
\phantomsection
\label{parameters:save_initial_mesh}

\index[prmindex]{save\_initial\_mesh}
\index[prmindexfull]{Output!save initial mesh}

{\it Default:} false (0)

{\it Description:} [Advanced] Determines whether the initial computational mesh should be saved for further debugging or not. Can be used to investigate the mesh refinement if it is asked by setting {\tt refine\_initial\_mesh} to true.

{\it Possible values:} Any boolean value (1 or 0)


\item {\it Parameter name:} {\tt save\_initial\_partitioned\_mesh}
\phantomsection
\label{parameters:save_initial_partitioned_mesh}

\index[prmindex]{save\_initial\_partitioned\_mesh}
\index[prmindexfull]{Output!save initial partitioned mesh}

{\it Default:} false (0)

{\it Description:} [Advanced] Determines whether the partitioned computational mesh should be saved for further debugging or not. Can be used to debug and view the output of the mesh partitioning process before going for the actual simulation.

{\it Possible values:} Any boolean value (1 or 0)


\end{itemize}
